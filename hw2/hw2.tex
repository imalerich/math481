\documentclass{jhwhw}
\author{Ian Malerich}
\title{Math 481: Homework 2}

\usepackage{amssymb, amsfonts, mathtools, graphicx, breqn, minted, subfig, float}
\usemintedstyle{friendly}
\graphicspath{ {png/} }

\begin{document}
\raggedright

\problem{}

    Find the characteristic equation and reduced characteristic equation of the 3-step method
    $$
	y_{n+1} = -\frac{27}{11}y_n + \frac{27}{11}y_{n-1} + y_{n-2} + 
	h\biggr[\frac{3}{11}f_{n+1} + \frac{27}{11}f_n + \frac{27}{11}f_{n-1} + \frac{2}{11}f_{n-2}\biggr]
    $$

    Find the (exact) roots of the reduced characteristic equation, and determine if this method
    is stable for small $h\lambda$ or not.

\solution
    
    First let us rewrite the method as
    $$
	y_{n+1} + \frac{27}{11}y_n - \frac{27}{11}y_{n-1} - y_{n-2} =
	h\biggr[\frac{3}{11}f_{n+1} + \frac{27}{11}f_n + \frac{27}{11}f_{n-1} + \frac{2}{11}f_{n-2}\biggr]
    $$

    Note that for this method we have 
    $$
	a_0 = -1, a_1 = -\frac{27}{11}, a_2 = \frac{27}{11}
    $$

    Which produces the characteristic polynomial
    \begin{align*}
	p(z) &= z^3 + \frac{27}{11}z^2 - \frac{27}{11} - 1 \\
	&= \frac{1}{11}(z-1)(11z^2 + 38z+11) \\
	&= \frac{1}{11}(z-1)
	    (z + \frac{1}{11}(19 + 4\sqrt{15}))
	    (z - \frac{1}{11}(4\sqrt{15} - 19))
    \end{align*}

    From this we find the zeros for the characteristic polynomial
    $$
	z = 1,
	z = -\frac{1}{11}(19 + 4\sqrt{15}),
	z = \frac{1}{11}(4\sqrt{15} - 19)
    $$

    Note that $|-\frac{1}{11}(19 + 4\sqrt{15})| \approx 3.1356303 \textgreater 1$. \\
    This implies that this method is not stable for small values of $h\lambda$.

\problem{}

    The IVP
    \begin{align*}
	y' &= -y + 2e^{-t}\cos(2t) \\
	y(0) &= 0
    \end{align*}
    has the true solution $y(t) = e^{-t}\sin(2t)$. \\
    Solve it numerically from t=0 to t=5, using
    \begin{enumerate}
	\item the classical 4-stage Runge-Kutta method with h=0.2
	\item a PECE predictor-corrector method based on 4-step AB, 3-step AM, h=0.2 \\
	    Use RK values from part (a) for the startup.
	\item the Matlab built-in ODE45. This routine will pick its own steps.
    \end{enumerate}

\solution
    \begin{center}
	\includegraphics[scale=0.95]{p1_0_2}

	\begin{tabular}[pos]{|l|r|r|r|r|}
	    \hline
	    & Actual	& RK4			& PECE - AB4/AM3	& ODE45 (h unknown) \\ \hline
	    h = 0.2 &&&& \\ \hline
	    Value	& -0.013911	& -0.013911		& -0.013776  & -0.014068 \\ \hline
	    Error	& 		& $2.1822*10^{-7}$	& $1.3560*10^{-4}$ & $1.5663*10^{-4}$\\ \hline
	    h = 0.1 &&&& \\ \hline
	    Value	& -0.013911	& -0.013911		& -0.013908 & \\ \hline
	    Error	& 		& $1.2907*10^{-8}$	& $3.3871*10^{-6}$ & \\ \hline
	\end{tabular}
    \end{center}

    \inputminted[linenos,frame=lines,framesep=2mm]{octave}{p2.m}

\problem{}

    Use the Matlab routine $fzero$ or something comparable to find the two solutions of
    $$ e^x - x - 2 = 0.$$
    There is one positive and one negative solution.

\solution
    
    Zeros exist at the points x = -1.8414 and x = 1.1462.
    \inputminted[linenos,frame=lines,framesep=2mm]{octave}{p3.m}

\problem{}

    Solve the boundary value problem
    \begin{align*}
	y''(t) &= \frac{3}{2}y^2(t) \\
	y(0) &= 4 \\
	y(1) &= 1 \\
    \end{align*}
    by a shooting method.

\solution

    \begin{figure}[H]
	\centering
	\subfloat[s $\in\lbrack -40, -5\rbrack$]{{
	    \includegraphics[scale=0.48]{p4_all} 
	}}
	\subfloat[Solutions]{{
	    \includegraphics[scale=0.48]{p4_sol}
	}}
    \end{figure}

    Using fzero to find zeros of the function phi(s) produces 
    values of s = -35.859 and s = -8.
    Figure (a) shows the solutions for integer s values in the interval
    $\lbrack -40, -5\rbrack$, where figure (b) displays only the solutions
    for s = -35.859 and s = -8 that satisfy the IVP with y(0) = 4 and y(1) = 1.
    The code to produce these solutions may be found below.

    \clearpage
    \inputminted[linenos,frame=lines,framesep=2mm]{octave}{p4.m}
    \inputminted[linenos,frame=lines,framesep=2mm]{octave}{phi.m}

\problem{}

    \begin{enumerate}
	\item This is a stiff ODE. Solev the equations on the interval [0, 0.001],
	    using a stiff solver such as ode15s. Plot the result.
	\item Answer the following questions.
	    \begin{itemize}
		\item How many steps did the stiff solver take? How much computer time?
		\item How many steps does a non-stiff solver, such as ode113 take? How
		    much computer time does it take?
		\item What is the clock period?
	    \end{itemize}
    \end{enumerate}

\solution

    \begin{center}
	\includegraphics[scale=1.0]{p5a}
    \end{center}

    From the code included below, we can tell that ode15s took 393 steps and 
    solved the stiff ODE in 0.721958 seconds whereas ode113 took 19484 steps to solve the stiff ODE in 1.1636209 seconds.
    The graph gives us a clock width of approximately
    $(8.25 * 10^{-4} - 4.45 * 10^{-4}) = 0.00038$ seconds.
    Thus the clock period is approximately 0.00076 seconds.

    \clearpage
    \inputminted[linenos,frame=lines,framesep=2mm]{octave}{p5.m}

\end{document}
